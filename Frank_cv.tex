\documentclass[8pt]{extarticle}
\pagestyle{empty}
\usepackage{geometry}
\geometry{
	paper=a4paper,
	top=0.75cm,
	bottom=0.5cm,
	left=.8cm,
	right=.8cm,
}

\usepackage{suffix}
\usepackage{xifthen}

\usepackage[sfdefault]{FiraSans}
\usepackage[hidelinks, pdftex, pdftitle={Curriculum Vitae – Maurice Frank}, pdfauthor={Maurice Frank}]{hyperref}
\usepackage{fontawesome}
\usepackage{xparse}

\usepackage{enumitem}
\setlist{left=4pt, topsep=5pt, partopsep=0pt, parsep=0pt, itemsep=4pt}

\usepackage{multicol}
\setlength\columnsep{.8cm}

\setlength{\parindent}{0mm}
\renewcommand{\baselinestretch}{1.1}
\frenchspacing

\usepackage{tikz}
\definecolor{primary}{rgb}{0.455, 0.635, 0.365}


% Command to vertically centre adjacent content
\newcommand{\vcenteredhbox}[1]{% The only parameter is for the content to centre
	\begingroup%
		\setbox0=\hbox{#1}\parbox{\wd0}{\box0}%
	\endgroup%
}

\newcommand{\icon}[3]{% The first parameter is the FontAwesome icon name, the second is the box size and the third is the text
	\vcenteredhbox{\colorbox{primary}{\makebox(#2, #2){\textcolor{white}{\LARGE\csname fa#1\endcsname}}}}% Icon and box
	\hspace{0.2cm}% Whitespace
	\vcenteredhbox{\large\textcolor{primary}{#3}}% Text
}

\newcommand{\bicon}[2]{
    \textbf{%
    \vcenteredhbox{\makebox(11, 11){\small\csname fa#1\endcsname}}%
    \hspace{0.05cm}%
    \vcenteredhbox{#2}}%
}

\newcommand*{\cvsect}[1]{
    \vspace{\baselineskip}
    \colorbox{primary}{\textcolor{white}{\MakeUppercase{\textbf{\large #1}}}}%
    \vspace{0.2\baselineskip}\\%
}

\WithSuffix\newcommand\cvsect*[1]{\cvsect{#1}\vspace{-0.8\baselineskip}}

\newcommand{\entry}[5]{%
    \filbreak%
    {\large #2}\vspace{0.15\baselineskip}\\%
    \bicon{MapMarker}{#3}\hfill%
    \bicon{Calendar}{#1}\vspace{.5\baselineskip}\\%
    #4\vspace{.5\baselineskip}\\%
    \ifthenelse{\isempty{#5}}{}{{\texttt{#5}}\vspace{.5\baselineskip}\\}%
}%

\renewcommand{\|}{\hspace{1.2mm}/\hspace{1.2mm}}

\begin{document}
\begin{minipage}[t]{0.45\textwidth}
    \vspace{-5pt}%
    {\Huge Maurice Frank}\\

    {\Huge\textcolor{gray}{Machine learning researcher}}
\end{minipage}
\begin{minipage}[t]{.275\textwidth}
        \icon{MapMarker}{12}{Amsterdam, NL}\\
        \icon{Phone}{12}{\href{tel:0031638493722}{+31 6 38493722}}\\
        \icon{At}{12}{\href{mailto:hello@morris-frank.dev}{hello@morris-frank.dev}}\\
\end{minipage}
\begin{minipage}[t]{.275\textwidth}
	\icon{User}{12}{\href{https://morris-frank.dev}{morris-frank.dev}}\\
    \icon{Github}{12}{\href{https://github.com/morris-frank}{github.com/morris-frank}}\\
    \icon{Linkedin}{12}{\href{https://linkedin.com/in/morris-frank}{linkedin.com/in/morris-frank}}\\
\end{minipage}
\vspace{5pt}

\begin{multicols*}{2}
\cvsect{Interests}

I like to hunt creative solutions to challenging problems with the use of machine learning.
I want my work to have a real positive impact on the world and I favour team-work with enthusiastic colleagues.
My current academic interest lies in deep generative modelling and Bayesian methods, VAEs and Flows.
I am especially intrigued by work on and applications to music and sound as well as an understanding of human perception.
I have strong theoretical backgrounds on machine learning and a background in physics and mathematics.\\\\

\cvsect{Education}
\entry{2018 -- 2020}
{M.Sc. Artificial Intelligence\hfill\textit{\normalsize GPA 8.8/10 \small{(est)}}}
{\href{https://uva.nl}{Universiteit van Amsterdam}}
{
    Research degree with most contemporary topics in AI/ML
    \begin{itemize}%
        \item Theory and practical implementation of the concepts of machine and deep learning: CNNs, RNNs/LSTMs, VAEs/Flows, GANs, MCMC/sampling, Gaussian process/Bayesian modelling, ICA,
        \item Specialization in topics: Theoretical machine learning, Computer Vision, Reinforcement Learning, NLP, Information theory
    \end{itemize}
    \textbf{Thesis:} Investigating Bayesian methods for un- and semi-supervised music source separation. Do generative models know what they don't know?\\
    \emph{Supervisor:} \href{https://scholar.google.com/citations?user=KNJIRGkAAAAJ}{Maximilian Ilse (AMLab)}
}
{}

\entry{2014 -- 2017}
{B.Sc. Applied Computer Science\hfill\textit{\normalsize GPA 3.48/4}}
{\href{https://uni-heidelberg.de/en}{University Heidelberg}}
{
    General degree on all topics of Computer Science
    \begin{itemize}
        \item Specialization on image processing and pattern recognition: Deep Learning, 3D Computer vision, medical image analysis, fairness in AI
        \item Courses on Practical CS, Algorithms, DBMS, Theoretical informatics, Numerical methods and software engineering
        \item Minor in the basic concepts of Physics
    \end{itemize}
    \textbf{Thesis:} Using a FCN-ResNet based detector the thesis provides a reverse image search tool here in particular to retrieve art historic images containing an object given by a reference image.\href{https://github.com/morris-frank/ba_latex/blob/master/thesis.pdf}{\bicon{Link}{Read}}\\
    \emph{Supervisor:} \href{https://scholar.google.com/citations?user=ZrRs-qoAAAAJ}{Dr\ Miguel Bautista Martin}, \href{https://hci.iwr.uni-heidelberg.de/Staff/bommer}{Prof.\ Dr\ Björn Ommer}
}
{}

\entry{2013 -- 2014}
{B.Sc. Physics}
{\href{https://uni-heidelberg.de/en}{University Heidelberg}}
{Change of degree after the first year, kept as minor studies.}
{}

\cvsect{Experience}
\entry%
{01/2020 -- 04/2020}
{Graduate Teaching Assistant}
{\href{https://uva.nl/}{Universiteit van Amsterdam}}
{
    Assisting with the master level courses:
    \begin{itemize}
        \item Information Retrieval
        \item Fairness, Accountability, Confidentiality and Transparency in AI
    \end{itemize}
    Each teaching tutorial sessions, correcting course work and support students in finishing scientific work and writing.
}
{Python{\|}PyTorch}

\entry%
{06/2019 -- 08/2019}
{Internship --- ML for production quality control}
{\href{https://bmwgroup.com}{BMW Group, Munich}}
{
    Worked within the engine development department.
    Building statistical models to analyze engine part quality and live fleet monitoring.
    Building a production-ready data visualization app.
}
{Python{\|}PySpark{\|}Palantir Foundry{\|}PostgreSQL{\|}Dash{\|}Agile development}

\entry%
{04/2018 -- 08/2018}
{Full-stack Web developer}
{\href{https://buergerwerke.de/}{Bürgerwerke eG, Heidelberg}}
{
    Ground-up development of a team communication and organization web portal in Ruby on Rails.
    Backend and Frontend work.
    Software engineering from planning the project until complete delivery.
}
{Rails{\|}Ruby{\|}Software Engineering}

\entry%
{09/2016 -- 06/2017}
{Laboratory admin}
{\href{https://uni-heidelberg.de/fakultaeten/wiso/awi/index_en.html}{Alfred-Weber-Institute, Heidelberg}}
{
    Administrator in a computer lab for Behavioral Economics experiments.
    Development of an experiment administration software.
}
{Ruby{\|}Python}

\entry%
{10/2015 -- 02/2016}
{Teaching Assistant}
{\href{https://uni-heidelberg.de/en}{University Heidelberg}}
{
    Assisting with the Bachelor course \textit{Practical Computer Science}.
    Teaching in self-prepared weekly tutorial session.
}
{C++{\|}C}

\cvsect{Voluntary work}
\entry%
{06/2015--}
{CO-Founder}
{\href{https://collegiumacademicum.de}{Collegium Academicum, Heidelberg}}
{
    Student-founded non-profit company building sustainable student housing.
    We are building an innovative living space for 200 young people.
    Creating an educational center for holistic self-learning.
}
{team building{\|}lead generation{\|}design work{\|}writing grant applications}

\cvsect*{Skills}
\begin{itemize}
    \item Excellent skills in machine learning engineering with Python: PyTorch, TensorFlow, Caffe, Pandas, SciPy, OpenCV, Scikit-Learn
    \item Excellent skills in using GNU/Linux systems: System administration, BASH, Perl, Git, Slurm
    \item Strong academic and technical writing skills as well as enthusiasm for writing useful tutorials (see blog)
    \item Strong skills in data visualization: Dash (plotly), D3, web technologies
    \item Engineering experience in working with: MATLAB, Java, C++, JavaScript
\end{itemize}

\begin{minipage}[t]{.5\columnwidth}
    \cvsect*{Languages}
    \begin{itemize}[left=3.5em]
        \item[\textbf{German}] native
        \item[\textbf{English}] fluent (TOEFL 112/120)
        \item[\textbf{Persian}] \textit{beginner}
    \end{itemize}
\end{minipage}
\begin{minipage}[t]{.5\columnwidth}
    \cvsect*{Personal interests}
    \begin{itemize}
        \item Climbing
        \item Mountaineering
        \item Electronic music production
    \end{itemize}
\end{minipage}

\end{multicols*}
\end{document}
